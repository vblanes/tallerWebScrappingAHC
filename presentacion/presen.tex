%presentacion
\documentclass{beamer}
\usepackage[utf8]{inputenc}
\usepackage{hyperref}

%tema de las trapas
\usetheme{Warsaw}
\usecolortheme{crane}


\title{Taller Web Scrapping con Python}
\subtitle{ÀREA HACKERS CÍVICS}
\author{Vicent Blanes}
\date{01/05/2017}
%comienzo documento
\begin{document}
	%empezamos con el primer frame
	\begin{frame}
		\titlepage
	\end{frame}

	%primera traspa
	\begin{frame}{¿Por qué Python?}
		\begin{columns}[T]
			\begin{column}[T]{8cm} 
				%lista de puntos
				\begin{itemize}  
					\item Python es un lenguaje \textbf{interpretado} y con una gran expresividad.
					\item Dispone de librerías potentes que nos ahorran mucho trabajo.
					\item Además es muy limpio y fácil de leer, un lenguaje ideal para aprender a programar.
				\end{itemize}
			\end{column}
		
			\begin{column}[T]{5cm}
				\includegraphics[height=3cm]{python.png}
			\end{column}
		\end{columns}
	\end{frame}
	%segunda traspa
	\begin{frame}{Herramientas para Web Scrapping}
		\begin{itemize}
			\item Utilizaremos la librería \textbf{urllib} para conectarnos a una dirección de Internet y obtener su contenido.
			\item Después utilizaremos \textbf{beautifulsoup4} para \textit{parsear} el contenido y poder extraer información de forma
			sencilla
			\item También utilizaremos la biblioteca \textbf{iPython} para trabajar con notebooks, que nos permite ejecutar fragmentos de 
			código de forma independiente y resulta mucho más interactivo.(?)
		\end{itemize}
	\end{frame}
	%tercera traspa
	\begin{frame}{Anaconda}
		\begin{columns}[T]
			\begin{column}[T]{8cm} 
				%lista de puntos
				\begin{itemize}  
					\item Anaconda es una distribución de python que tiene pre-instaladas montones de librerías útiles 
					\item Su instalación es muy sencilla. Podéis encontrarla en \textbf{\url{https://www.continuum.io/downloads}}.
					\item Además posee su propio gestor de paquetes. Si te falta alguna librería puedes usar \textbf{conda install} 
					y el nombre de esta.
					\item Para el uso de los \textit{notebooks} tenemos que ejecutar \textbf{conda install ipython}
				\end{itemize}
			\end{column}
			
			\begin{column}[T]{5cm}
				\includegraphics[height=2cm]{anaconda.png}
			\end{column}
		\end{columns}
	\end{frame}


\end{document}