\documentclass{article}

\usepackage[utf8]{inputenc}
\usepackage[spanish, es-tabla]{babel}
\usepackage{ragged2e} % Para justificar texto
\usepackage{listings}
\usepackage{graphicx}
\usepackage[autostyle]{csquotes}
\usepackage{amsmath}
\usepackage{color}
\usepackage[colorlinks = true,
linkcolor = black,
urlcolor  = blue,
citecolor = blue,
anchorcolor = black]{hyperref}


\definecolor{codegreen}{rgb}{0,0.6,0}
\definecolor{codegray}{rgb}{0.5,0.5,0.5}
\definecolor{codepurple}{rgb}{0.58,0,0.82}
\definecolor{backcolour}{rgb}{0.95,0.95,0.92}



\lstdefinestyle{mystyle}{
    backgroundcolor=\color{backcolour},
    commentstyle=\color{codegreen},
    keywordstyle=\color{magenta},
    numberstyle=\tiny\color{codegray},
    stringstyle=\color{codepurple},
    basicstyle=\footnotesize,
    breakatwhitespace=false,
    breaklines=true,
    captionpos=b,
    keepspaces=true,
    numbers=left,
    numbersep=5pt,
    showspaces=false,
    showstringspaces=false,
    showtabs=false,
    tabsize=2
}

\lstset{style=mystyle}

\title{Anexo funciones Python - Taller de Web Scrapping}
\date{20/02/2017}
\author{Vicent Blanes Selva}
\begin{document}
\maketitle
\newpage
\tableofcontents
\newpage
\section{Estructuras}
\justify
En esta sección haremos un recordatorio/pequeña introducción de la sintaxis 
en Python. Si estás familiarizado con otros lenguajes de programación como C, Java, JavaScript o R 
puedes leerla por encima para no cometer errores de sintácticos. Si no estás familiarizado con la programación
te recomendamos que le eches un buen ojo ;)\\\\
Enlace directo a la documentación \textbf{\url{https://docs.python.org/3.6/tutorial/datastructures.html}}
	\subsection{Estructuras de datos}
		\subsubsection{Listas}
			\justify
			Una lista es una estructura muy sencilla que consiste en una colección
			de elementos almacenados de forma consecutiva. A diferencia de lo que ocurre en otros lenguajes
			python puede contener elementos de distintos tipos. Se puede definir una lista vacía o 
			con algunos objetos ya en su interior. 
			\lstinputlisting[language=python, firstline=1, lastline=9]{minicodigos/lista.py}
			\justify
			La forma más sencilla de recuperar un elemento de la lista es referirse a él mediante
			el nombre de la lista y el índice que ocupa entre corchetes. \textbf{IMPORTANTE:} en python 
			el primer indice de una lista es el 0.
			\lstinputlisting[language=python, firstline=11, lastline=15]{minicodigos/lista.py}
			\justify
			También poseemos dos método para insertar elementos en una lista, \textbf{append e insert}.
			\lstinputlisting[language=python, firstline=18, lastline=26]{minicodigos/lista.py}
			\justify
			Y entre otros, un par de método para borrar los elementos de una lista. Se recomienda
			revisar la documentación (enlace al principio de la sección).
			\lstinputlisting[language=python, firstline=28]{minicodigos/lista.py}
		\subsubsection{Diccionarios}
			\justify
			Los diccionarios son estructuras de datos similares a las listas, sólo que en estos se guardan pares clave-valor. 
			Cada una de las entradas debe poseer una clave única, pudiendo repetirse su valor. También son conocidos como 
			\textbf{Maps} o con implementaciones concretas como \textbf{tablas hash}. 
			Poseemos métodos para insertar, borrar y consultar las información de los diccionarios. Lo vemos con un ejemplo:
			\lstinputlisting[language=python, lastline=17]{minicodigos/diccionario.py}
			\justify
			Tenemos también método para recuperar en forma de lista las clave y los valores. 
			Debemos destacar que al contrario de las listas, los diccionarios en python no tienen un orden y por tanto cada vez que
			recuperemos las claves y/o los valores van a tener un orden arbitrario que nada tiene que ver con el orden de inserción.
			\lstinputlisting[language=python, firstline=20]{minicodigos/diccionario.py}
		
	\subsection{Estructuras de control}  
		\subsubsection{Condicional: if}
		\justify
		La clausula if, seguida de una condición que se evalúa a verdadero o falso y dos puntos, dan paso a un bloque de código
		que sólo se ejecutará si la mencionada condición es verdadera. La mejor forma de verlo es con un ejemplo:
		\lstinputlisting[language=python]{minicodigos/ejemploif.py}
		\justify
		\textit{Nota: los bloques de código en python están marcados por la identación. \textbf{El estándar son 4 espacios} así 
		que se recomienda configurar tu editor favorito para que sustituya la inserción de tabuladores por la de 4 espacios}.
		\subsubsection{Bucle: for}
		\justify
		Al igual que los condicionales, los bucles son un parte muy importante de la programación ya que permiten
		realizar un conjunto de acciones varias veces sin tener que repetir el código. Los usaremos principalmente para recorrer 
		elementos de una lista y poder ejecutar acciones sobre ellos. Los bucles for en python son algo distintos a los 
		bucles de otros lenguajes más clásicos. De hecho serían el equivalente a un bucle \textbf{for each} tradicional.
		\lstinputlisting[language=python]{minicodigos/ejemplofor.py}
	 
\section{Funciones python}
	\justify
	En este apartado veremos algunas funciones útiles que posee 
	python en una distribución sin librerías externas y que nos ayudarán
	a desarrollar nuestro código.
	\begin{itemize}
		\item \textbf{print}: Función que imprime por la salida estándar.
		\item \textbf{str(obj)/int(obj)/float(obj)}: Devuelve, si es posible, el objeto transformado en cadena de 
		texto, entero o número de coma flotante dependiendo de la función utilizada. Puede producir error.
		\item \textbf{len(obj)} Devuelve la longuitud de una colección, como el tamaño de una lista o el número 
		de clave de un diccionario. Produce error si se aplica sobre variables simples como un número o un literal.
		\item \textbf{range(numero)}: Devuelve una secuencia de números enteros desde 0 hasta el número especificado
		sin incluirlo (Intervalo abierto por la derecha). Se pueden añadir más argumentos, como el número de inicio 
		o el tamaño del paso.\\
		\begin{itemize}
			\item range(4) devuelve [0,1,2,3]
			\item range(2,4) devuelve [2,3]
			\item range(1,4,2) devuelve [1,3] \#de uno hasta 4 yendo de 2 en 2
		\end{itemize}
	\end{itemize}
\section{Funciones de urllib}
	\justify
	La librería urllib es un conjunto de paquetes diseñados para hacer más fácil las interacciones a partir de peticiones web.
	Nos vamos a centrar en \textbf{urllib.request} pero es una librería muy útil y recomendamos revisar su documentación (
	\textbf{\url{https://docs.python.org/3/library/urllib.html}}). Podemos ver que se ofrecen montones de funciones para
	realizar peticiones web de modo muy avanzado, su exploración concreta queda fuera del taller pues nosotros únicamente
	centraremos en aquellas que nos permitan hacer un \textit{scrapeado} eficiente sin entrar en temas de \textit{proxies} o identificación.
	\begin{itemize} 
		\item \textbf{urlopen(url)} se trata de una función para establecer conexión con un sitio web, se pueden
		especificar algunos argumentos opcionales más pero en nuestro caso sólo la usaremos con la página web 
		a la que deseemos conectar.
		\item Sobre el objeto retornado por la función anterior se puede aplicar la función \textbf{read()} que 
		nos devuelve el contenido web de la página a la que hayamos realizado la petición como una única cadena de
		texto.
	\end{itemize}
	
\section{Funciones de beautifulsoup4}
	\justify
	BeautifulSoup es una librería que permite tratar documentos html de cualquier tamaño para poder
	hacerlo más legible o realizar consultas como por ejemplo, obtener el primer texto que esté entre las etiquetas $<p>.$
	Aquí puedes encontrar su documentación con numerosos ejemplos: \textbf{\url{https://www.crummy.com/software/BeautifulSoup/bs4/doc/}}
	\begin{itemize}
		\item \textbf{BeautifulSoup(cadenahtml, 'lxml')} esta es la función principal que transforma una cadena de texto
		con información html en un objeto del tipo BeautifulSoup que representa información anidada y posee varios métodos
		para tratar la información.
	\end{itemize}
	
\section{Para más información...}
\justify 
Este pequeño anexo no es más que una ayuda para entender algunos de los muchos elementos 
del lenguaje que vamos a usar en este taller,
si se quiere aprender más en profundidad recomendamos acudir a fuentes bibliográficas. 
Nos gustaría recomendar el libro \textbf{Python para todos}, que a pesar de estar enfocado a Python 2.7 
es libre y muy didáctico. Se puede descargar en este enlance: \textbf{\url{https://launchpadlibrarian.net/18980633/Python\%20para\%20todos.pdf}}



%\lstinputlisting[language=R]{graficas.R}
%\includegraphics[scale=0.4]{angulo_vicent_eje_electrico}

\end{document}